\documentclass{scrreprt}

%%% Sane encodings %%%%%%%%%%%%%%%%%%%%%%%%%%%%%%%%%%%%%%%%%%%%%%%%%%%%%%%%%%%%%
\usepackage[utf8]{inputenc}
\usepackage[T1]{fontenc}

%%% Packages %%%%%%%%%%%%%%%%%%%%%%%%%%%%%%%%%%%%%%%%%%%%%%%%%%%%%%%%%%%%%%%%%%%
% We're writing in English
\usepackage[british]{babel}

% Support PDF properly
\usepackage{hyperref}
\hypersetup{
	% Do not draw boxes around links, colour them
	colorlinks,
	% Encode PDF strings in Unicode, not ASCII
	unicode,
}

% Reducing easily detectable mistakes
\usepackage[l2tabu, orthodox, experimental]{nag}

% Better typography
\usepackage{microtype}

% More control over list environments
\usepackage{enumitem}

% Additional symbols not otherwise available in T1
\usepackage{textcomp}

%%% Looks %%%%%%%%%%%%%%%%%%%%%%%%%%%%%%%%%%%%%%%%%%%%%%%%%%%%%%%%%%%%%%%%%%%%%%
% Do number chapters but not anything beneath
\setcounter{secnumdepth}{0}

% Display chapters and sections in TOC but not anything beneath
\setcounter{tocdepth}{1}

% Mark up file ending consistently
\newcommand*\fileending[1]{\texorpdfstring{\kern-.16em\texttt{#1}}{#1}}

% Entries in menus and navigations
\newcommand*\menu[1]{\textsf{\textbf{#1}}}
\makeatletter
\newenvironment*{menudescription}{%
	\newcommand*\entry[1]{\item[\menu{##1}]}
	\begin{description}
}{%
	\end{description}
}
\makeatother

% Usage: \Path{a}; \Path[a][b][c]{d}
% Result: a; a -> b -> c -> d
\makeatletter
\def\Path{\@ifnextchar[{\@with}{}}
\def\@with[#1]{#1\allowbreak\thinspace\textrightarrow\thinspace\Path}
\makeatother

% Markup for things that do not have proper markup yet but might want some
\newcommand*\todomu[1]{\textcolor{blue}{#1}}

%%% Metadata %%%%%%%%%%%%%%%%%%%%%%%%%%%%%%%%%%%%%%%%%%%%%%%%%%%%%%%%%%%%%%%%%%%
\hypersetup{pdftitle="UltraStar Deluxe"}
\title{UltraStar Deluxe}
\subtitle{User Manual}
\author{P A Harsent, canni, dppes}
\date{2019-12-23}

\begin{document}
\maketitle

\tableofcontents

\chapter{Introduction}

\chapter{Installation}

\section{Windows}

\section{Linux}

\section{Mac}

\chapter{Building}

\chapter{Configuration}

Most configuration can be accessed via
\Path[\menu{Tools}]{\menu{Options}},
which provides us with several sub-pages:
\begin{menudescription}
\entry{Game}
	General game settings.
\entry{Graphics}
\entry{Sound}
\entry{Input}
\entry{Lyrics}
\entry{Themes}
	Change the look of UltraStar Deluxe
	by choosing themes, skins and colours.
\entry{Record}
\entry{Advanced}
\entry{Network}
\entry{Webcam}
\entry{Jukebox}
\end{menudescription}

\section{Game}

Here, you can configure general settings of the game.

\begin{menudescription}
\entry{Language}
	The language in which the game is played.
	Contrary to what the in-game help claims,
	the changes go into effect immediately.
\entry{Song Menu}
	The type of menu used to display the available songs
	when choosing what to sing next.
\entry{Tabs}
\entry{Sorting}
\entry{Show Scores}
\entry{Debug}
\entry{Audio/Video Delay}
\entry{Microphone Delay}
\end{menudescription}

\section{Graphics}

\begin{menudescription}
\entry{Fullscreen}
	Disable full screen mode or choose one of two full screen modes.
	This requires a restart,
	other than the full screen hot key
	that immediately toggles between windowed and full screen mode.
\entry{Resolution}
	When \menu{Fullscreen} is \todomu{Off},
	this configures the size of the window.
	When \menu{Fullscreen} is \todomu{On},
	it does weird stuff.
	When \menu{Fullscreen} is \todomu{Borderless},
	it is unavailable.
\entry{Depth}
	The depth option changes the number of colours
	that are used within UltraStar Deluxe.
	Allowed options are:
	\todomu{16 bit} (65\,536 colours)
	and \todomu{32 bit} (4\,294\,967\,296 colours).
	On lower powered systems,
	the higher the colour depth, the slower the performance.
	So if UltraStar Deluxe does not run smoothly,
	change this to \todomu{16 bit}.
\entry{Visualization}
\entry{Oszilloscope}
	The Oscilloscope setting reacts to the microphone input
	and is useful for setting up new songs.
	Therefore leave this option set to \todomu{Off}
	unless a new song is being set up.
	For more information on setting up new songs, see \ref{new_song}.
\entry{Movie size}
\end{menudescription}

\chapter{Customization}

\chapter{Statistics}

\chapter{Importing Songs}

One of the big advantages of UltraStar Deluxe
is the ability to add new songs.
A song contains the music in an \fileending{.mp3} file
and the lyrics and other settings in a \fileending{.txt} file,
known as a lyric file.
For legal copyright reasons,
the \fileending{.mp3} music file cannot be copied from person to person,
but a new \fileending{.txt} file usually can,
as long as the creator of the \fileending{.txt} gives permission, which most do.

\section{Obtaining lyrics files}
New \fileending{.txt} lyrics files are available
and can be downloaded from the UltraStar Deluxe Forums
and are then matched to an \fileending{.mp3} file that you already own.

\section{Creating new lyrics files}

\section{Adding a Song}
\begin{itemize}
\item Ensure that UltraStar Deluxe is not running
\item Under the \todomu{Songs} folder,
	create a new folder with the name of the artist and song title,
	e.g.\@ \todomu{Abba – Dancing Queen (ABBA gold CD album track)}
\item Copy your \fileending{.mp3} file into the new folder
\item Save the lyrics file alongside the audio file in the same folder
\item In the new song folder,
	make sure that the \fileending{.mp3} audio file
	and the \fileending{.txt} lyric file have the same name
\end{itemize}

\chapter{Troubleshooting}

\section{New Song}

\label{new_song}

When a newly acquired song does not work as it should,
use notepad (or another text editor) to open the \fileending{.txt} lyric file
and perform the following checks:
\begin{itemize}
\item Make sure that the \todomu{\#TITLE}, \todomu{\#ARTIST}, \todomu{\#MP3},
	\todomu{\#BPM} and \todomu{\#GAP} fields
	are all there
	and that there is an \todomu{E}
	at the very end of the file.
\item Make sure that the \todomu{\#MP3} value
	holds the exact filename of the \fileending{.mp3} audio file.
\end{itemize}
If you made any changes,
do not forget to save the file afterwards before retrying.

\section{Performance}

\subsection{Song Selection}

When the song selection becomes sluggish,
try to chose a different \menu{Song Menu}
in \Path[\menu{Tools}][\menu{Options}]{\menu{Game}}.

\appendix

\chapter{Lyric Files: \fileending{.txt} Format}

\chapter{Lyric Files: \fileending{.xml} Format}

\chapter{Advanced Configuration: \fileending{.ini} Format}

\chapter{About this Manual}
\section{Legal}
\subsection{Disclaimer}
The information in this document is provided as is
and there is no warrant to its completeness or accuracy;
The developers of UltraStar Deluxe or any associated personnel
shall not be responsible for any errors, omissions, loss or damage
whether indirect, special, incidental or consequential
resulting from the use of the information that contained herein.

\subsection{Acknowledgements}
All trademarks are acknowledged.

\subsection{Usage}
This document is licensed for use
under the GNU Free Documentation License 1.3.3.
\url{http://www.gnu.org/copyleft/fdl.html}

\section{Versions}

\newenvironment*{version}[2]{%
	\item[#1] (by #2)
	\begin{itemize}[nosep, noitemsep]
}{%
	\end{itemize}
}
\begin{description}
\begin{version}{2010-09-24}{P A Harsent}
\item Creation of initial version
\end{version}
\begin{version}{2010-10-02}{P A Harsent}
\item Tidied ambiguity over channel numbers in two player sound set up
\item Added further solutions to common graphic card problems
\item Added further solutions for common lyric problems
\end{version}
\begin{version}{2010-10-09}{P A Harsent}
\item Added main index
\item Added further solutions to common graphic card problems
\item Added \#GAP section when importing new songs
\item Added further solutions for common lyric problems
\item Added section on support for UTF8 layouts
\item Added .RAW section
\item Added further Windows 7 \& Vista considerations
\item Added multiple song directories section
\item Added alternate cover location section
\end{version}
\begin{version}{2010-10-10}{canni}
\item Corrections in layout and updated information to fit changes in 1.1 final
\end{version}
\begin{version}{2019-12-23}{dppes}
\item Started \LaTeX\ migration
\end{version}
\end{description}

\chapter{Index}

\end{document}
